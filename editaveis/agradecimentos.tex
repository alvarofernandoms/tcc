\begin{agradecimentos}

Agradeço a Deus pelos dons e capacidades cognitivas. Ao meu orientador, Paulo
Roberto Miranda Meirelles por já em nossos primeiros encontros em sala de aula,
em 2013 (primeiro ano como professor da UnB-FGA), na disciplina de Técnicas de
Programação, enxergar em mim algum potencial e confiar desde o início a
participação em projetos que cito hoje como as minhas primeiras experiências
profissionais, que certamente me ajudaram e ainda irão produzir frutos. Agradeço
aos amigos Carlos Coêlho, Fagner Rodrigues, Leandro Alves, João Paulo Alves e
Yeltsin Suares que foram de fundamental apoio para minha formação, iniciaram o
curso comigo e até hoje estão próximos. Aos mestrandos Diego Araújo e Rafael
Manzo, pois sem as instruções e disponibilidade deles este trabalho de conclusão
não seria realizado. Aos vários colegas de classe que compartilharam os mais
diversos momentos de alegrias e frustrações, e que amigavelmente me apelidaram de
“Jesus”! São muitos os nomes, mas se em algum momento tivemos que virar uma
noite finalizando algum trabalho/projeto/estudo, saiba que foi uma honra e que
agradeço o compartilhamento mútuo de conhecimento e risadas. Aos colegas,
professores e profissionais seniores do Laboratório Avançado de Produção,
Pesquisa \& Inovação em Software (LAPPIS) pelas várias horas empenhadas em
construir software sem nos precipitarmos, tendo calma cinco vezes e não sendo
egoístas. Agradeço aos meus pais, José Fernando e Edna, e às minhas irmãs,
Angélica e Heloísa, que me apoiaram e me sustentaram incondicionalmente, e por
serem os únicos que me viram deitar mesmo no chão para tentar relaxar no final
de semestres difíceis. Agradecimento especial à minha namorada, Nicole Pachêco,
que foi a peça fundamental para a minha mudança na vida acadêmica, pela
compreensão e carinho, e por ter sido minha fiel espectadora nos ensaios para a
defesa deste trabalho e também fiel revisora gramatical (manja muito das
letras!). Por fim, aproveito para reforçar um agradecimento especial para minha
mãe, Edna Matos de Souza, e dedicar à ela este trabalho, pois sem ela não
conseguiria caminhar (figurativamente e literalmente) e com certeza não me
tornaria Engenheiro de Software. Muito obrigado!

\end{agradecimentos}

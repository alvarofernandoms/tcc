\begin{resumo}[Abstract]
 \begin{otherlanguage*}{english}
   The Mezuro is a free \textit{Web} platform evaluation of source code. Since
   metrics can be defined as measures of various features of a software, Mezuro
   evaluates the static metrics of projects developed in certain languages.
   These metrics are exposed through numbers and readings groups, exposing the
   user aspects of size and quality of their project by simply inform the URL of
   the remote version control system. The objective of this study is to perform
   an exploratory analysis of this tool, addressing aspects of visualization of
   the software code metrics analyzed by it. The following contributions are
   planned:  validation of Mezuro metrics settings and viewing software
   specification. As a scientific contribution, or research activity, it will
   be done the analysis of the projects available on the Portal of the Brazilian
   Public Software (SPB) chosen because there is a possibility of incorporating
   the Mezuro in this Portal. In addition to student involvement in the
   evolution of the SPB. Another contribution is the study of the possible
   software visualization techniques applied to Mezuro.

   \vspace{\onelineskip}

   \noindent
   \textbf{Key-words}: Software Engineering. Mezuro. Software visualization.
   Metrics for source code. Software Evolution. Brazilian Public Software.
   Brazilian Public Software.
 \end{otherlanguage*}
\end{resumo}

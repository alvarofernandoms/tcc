\begin{resumo}
 O Mezuro é uma plataforma \textit{Web} livre de avaliação de código-fonte.
 Considerando que métricas podem ser definidas como medidas de diversas
 características de um software, o Mezuro avalia as métricas estáticas de
 projetos desenvolvidos em certas linguagens. Essas métricas são expostas por
 meio de números e grupos de leituras, expondo ao usuário aspectos de tamanho e
 qualidade do seu projeto, bastando informar apenas a URL do sistema de controle
 de versão remoto.
 O objetivo deste trabalho é realizar uma análise exploratória desta ferramenta,
 abordando aspectos da visualização das métricas de código dos softwares
 analisados por ela.

 \vspace{\onelineskip}

 \noindent
 \textbf{Palavras-chaves}: Engenharia de Software. Mezuro.
 Métricas de código-fonte. Evolução de Software. Visualização de Software.
 Software Público Brasileiro.
\end{resumo}

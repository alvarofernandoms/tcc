\chapter{Análise Exploratória do Projeto Mezuro}

% TODO: add introdução com um brief da seleção dos projetos do SPB e quantificação

\section{Exemplo de Uso: Projetos do Software Público Brasileiro}

O Software Público Brasileiro (SPB) é uma inciativa do Governo Federal que visa
o compartilhamento de softwares, experiências e informações amparado pela tese
do bem público como aquele que apresenta características de indivisibilidade e
de não rivalidade, ou seja, que pode ser usado por todos sem que com isto se
estabeleça competição pelo bem entre os usuários. A disponibilização dos
softwares no SPB é justificada pelo caráter cada vez mais estratégico para
governos e sociedade, pela similaridade de demandas entre os órgãos e entidades
públicas, pela racionalização dos recursos humanos, materiais e de tecnologia
da informação para seu atendimento e pelo acervo de soluções desenvolvidas
pelos diferentes poderes e esferas governamentais \cite{santos2011in01}.

Este capítulo trata da análise das métricas do código-fonte dos projetos do SPB
via Mezuro.

O SPB portanto é uma plataforma \textit{Web} de colaboração integrada. É composto por um
conjunto de ferramentas. São elas \cite{aboutSPB}:

\begin{itemize}
  \item \textbf{Noosfero:} Plataforma \textit{Web} Livre de Redes Sociais, desenvolvida
	em Ruby on Rails. Responsável pelas funcionalidades de gerenciamento de
	conteúdo (CMS) do SPB, permitindo a criação de páginas de usuários, softwares
	e comunidades de forma flexível e customizável. É a ferramenta com maior
	integração com o usuário, pois é responsável pelas páginas dos softwares
	disponibilizados, gerenciamento dos downloads destes softwares, criação de
	notícias e categorização dos softwares;
  \item \textbf{Mailman:} É uma aplicação para gerenciamento de listas de
	e-mail, escrito majoritariamente em Python e faz parte do projeto GNU. Cada
	software disponibilizado no SPB possui sua própria lista de e-mail e ela é um
	dos meios de comunicação entre os administradores e usuários dos softwares.
	Também serve para divulgação de notícias ou relatórios de novas versões.
	Possibilita ao usuário a participação em tópicos de discussão respondendo
	através do próprio e-mail, sem a necessidade de acessar diretamente o SPB;
	\item \textbf{Gitlab:} Gerenciador de repositórios Git com funcionalidades de
	wiki e mapeamento de issues, escrito em Ruby;
	\item \textbf{Mezuro:} Plataforma \textit{Web} Livre para avaliação colaborativa de
	código fonte, desenvolvido em Ruby on Rails, também adicionada ao SPB para
	fornecer acompanhamento da qualidade do código dos projetos \cite{aboutSPB}.
\end{itemize}

A integração das ferramentas acima citadas é realizada por meio do
\textbf{Colab}, um software desenvolvido em Python e Django que suporta
ambientes colaborativos e agregados. Em resumo, é uma plataforma para
autenticação, buscas e inserção de conteúdo no SPB \cite{aboutSPB}.

% TODO: add figura da arquitetura
% https://softwarepublico.gov.br/doc/arquitetura.html

\section{Resultados}

Foi realizado uma categorização dos softwares disponibilizados no Portal do
Software Público Brasileiro, como insumo que corroborará com as hipóteses
levantadas neste trabalho. A adição dos softwares do SPB à ferramentas de
monitoramento (Mezuro e CodeClimate\footnote{\url{https://codeclimate.com/}})
foi planejada para que o código seja avaliado e para que haja um incentivo aos
desenvolvedores e administradores das comunidades do Portal ao monitoramento de
seus projetos. Esta adição ou contribuição foi registrada e está disponível
para acesso por meio deste trabalho ou na lista de discussão para
desenvolvedores do SPB.

Como explicado na seção anterior, uma das ferramentas do SPB é o Gitlab
(serviço para armazenamento de repositórios Git) e foi constatado que
aproximadamente 75\% dos softwares \textbf{não} possuem seus códigos-fonte
versionados nesta ferramenta. Realizado algumas pesquisas, foi encontrado o
código-fonte em outros serviços (Github, Bitbucket). Fato esse que já auxiliou
neste primeiro desafio de adicionar determinados softwares para avaliação no
Mezuro. Porém, se considerarmos o SPB como a plataforma de colaboração
integrada, colaboração que é um dos objetivos fundamentais ao Governo Federal
com o Portal, o infortúnio de ter aproximadamente três-quartos dos softwares
não versionados é algo que atinge toda a comunidade que utiliza o SPB: sejam
desenvolvedores interessados em contribuir/evoluir os softwares; sejam usuários
finais que também contribuem com registro de problemas, ou issues, ou bugs.

Por padrão, no momento da disponibilização de um novo software no SPB, já são
criadas "instâncias" de cada uma das ferramentas que compõem o Portal. Bastando
ao administrador adequar o código-fonte ao versionamento com o Git. Por exemplo
o software "xpto", teria o seguinte domínio para acesso ao código fonte:
softwarepublico.gov.br/gitlab/xpto/xpto.

Como solução, durante a categorização, foi realizado o "commit inicial" para
aqueles softwares que disponibilizaram o código-fonte por meio de arquivo
compactado. Como forma de centralização e organização, foi criado a organização
"spb-metrics" no Github\footnote{\url{https://github.com/spb-metrics}}, pois
para avaliação no CodeClimate, os repositórios precisam estar neste serviço e
serem públicos. De outra maneira, ou em outro serviço, a avaliação teria um
custo estipulado pelas regras de negócio do próprio CodeClimate. Já no Mezuro é
possível adicionar um repositório para avaliação independentemente do serviço
(Github, Gitlab, Bitbucket, etc), basta ser utilizado o sistema de controle de
versão Git ou SVN. Comparação esta que reforça o estudo da visualização de
software como possível auxílio e contribuição ao Mezuro.

A categorização completa dos softwares do SPB está no Anexo \ref{chap:anexoB}

As imagens \ref{fig:is_software_gitlab_spb} e \ref{fig:linguagens_softwares_spb}
mostram respectivamente o percentual de softwares com repositório Git no SPB e
em qual linguagem foram escritos predominantemente. Com 45\%, PHP é a linguagem
em que é desenvolvido a maioria dos softwares do Portal SPB. Seguido por Java,
com 18\% e outras (Python, C++, Perl), sendo menos de 5\% cada. Foi decidido
portanto iniciar a avaliação de softwares escritos em PHP e com código-fonte
disponibilizado (compactado ou versionado).

\begin{figure}[!htb]
	\centering
    \includegraphics[keepaspectratio=true,scale=0.5]
    {figuras/is_software_gitlab_spb.eps}
  \caption{Porcentagem de Softwares Versionados no Gitlab do SPB}
  \label{fig:is_software_gitlab_spb}
\end{figure}

\begin{figure}[!htb]
	\centering
    \includegraphics[keepaspectratio=true,scale=0.5]
    {figuras/linguagens_softwares_spb.eps}
  \caption{Linguagens em que os Softwares do SPB são desenvolvidos}
  \label{fig:linguagens_softwares_spb}
\end{figure}

\newpage

\section{CodeClimate}

A análise de projetos em PHP no CodeClimate é a junção da análise dos seguintes
motores de análises \cite{enginesCodeClimate}:

\begin{itemize}
  \item \textbf{CSSLint:} como esperado de uma ferramenta \textit{lint}, esse
	motor faz uma análise estática do código CSS e aponta padrões que podem ser
	erros ou que possivelmente podem causar problemas aos desenvolvedores;
	\item \textbf{Duplication:} este motor analisa e indica estruturas ou blocos
	de códigos semelhantes;
	\item \textbf{ESLint:} outra ferramenta \textit{lint}, porém para análise e
	verificação de estilo de códigos escritos em EcmaScript ou Javascript;
	\item \textbf{FIXME:} realiza uma busca sensível à caixa alta por palavras
	como \textbf{TODO, FIXME, HACK, BUG} e destaca essas informações reforçando a
	importância ao desenvolvedor, que provavelmente deve consertar alguma parte no
	código agora, não depois;
	\item \textbf{PHP Code Sniffer:} análise estática no código PHP que detecta
	violações de um padrão já definido e estabelecido pela comunidade;
	\item \textbf{PHP Mess Detector:} procura por diferentes potenciais problemas
	com o código escrito em PHP, como possíveis bugs, código desnecessário,
	expressões demasiadamente complicadas, e parâmetros ou métodos não utilizados.
\end{itemize}

A figura \ref{fig:is_codeclimate_php_success} mostra o percentual de sucesso na
análise dos projetos em PHP do SPB.

\begin{figure}[!htb]
	\centering
    \includegraphics[keepaspectratio=true,scale=0.6]
    {figuras/is_codeclimate_php_success.eps}
  \caption{Porcentagens de Softwares com Análise bem sucedida no CodeClimate}
  \label{fig:is_codeclimate_php_success}
\end{figure}

\newpage

Considerando estes motores e ferramentas utilizadas pelo CodeClimate, pode-se
observar que nem todo projeto pode ser analisado por um conjunto de métricas ou
definições de estilo pré estabelicida. Fato este que intensifica a utilização do
Mezuro em que o usuário é capaz de criar as Configurações com o conjunto de
métricas que desejar, adaptando assim ao seu projeto. Portanto, se não há um
conjunto fechado ideal para todo projeto, é possível afirmar que também não há
uma visualização única para todas as Configurações ou conjunto de métricas.

% TODO: tabelas com os resultados dos projetos
%   - Mezuro
%   - CodeClimate
%   - Comparação

\section{Discussão dos Resultados}

\subsection{Exemplo da Análise por Projeto: Xêmele}

% TODO: add figuras

O Xemelê\footnote{\url{https://softwarepublico.gov.br/social/xemele}} é um
software para o gerenciamento de ambientes integrados com sites, blogs, chats,
wikis e e-mails. Desenvolvido na linguagem de programação PHP e utilizando outros
recursos, como HTML, Javascript, JQuery e banco de dados MySQL.

A análise realizada pelo Mezuro desta ferramenta, retornou a nota zero para o
projeto como um todo. E a métrica de destaque, relacionada com o arquivo
\textbf{functions.php} da raíz do projeto, está relacionada com o método
\textbf{recent\_comments} em que é recomendado a remoção da expressão
\textit{else} da linha 52 e a simplificação do código para trabalhar sem esta
expressão de condição.

Link para a avaliação no Mezuro\footnote{\url{http://mezuro.org/pt/repositories/114}}.

Já a nota dada ao projeto pelo CodeClimate foi a de 2.18/4. E os arquivos com
notas \textbf{F} e \textbf{D} são os arquivos Javascript \textit{js/jquery.cycle.js}
e \textit{js/jquery.js}, respectivamente. Além disso, a ferramenta mostra para o
usuário 4 páginas de possíveis problemas encontrados, nas diversas categorias,
tais como, problemas com complexidade, duplicação de código, estilo nos arquivos
CSS, clareza, possíveis ou risco de bugs e problemas com compatibilidade. E o
mesmo destaque dado ao arquivo \textbf{functions.php} pelo Mezuro, também foi
destacado no CodeClimate, mesmo assim, foi atribuído a maior nota possível, nota
\textbf{A}. Por fim, é destacado que as \textit{engines} que apontaram estes possíveis
erros são: \textit{csslint}, \textit{duplication}, \textit{eslint} e
\textit{phpmd}

Link para a avaliação no CodeClimate\footnote{\url{https://codeclimate.com/github/spb-metrics/xemele}}.

\subsection{Discussão dos Resultados}

Foram adicionados 31 softwares do SPB em ambas as ferramentas (Mezuro e
CodeClimate), desenvolvidos em PHP e Python. Estas adições resultaram na
análise descrita nos próximos parágrafos.

No Mezuro, dos 31 softwares adicionados, somente 4 obtiveram sucesso na
avaliação. Alguns dos problemas encontrados são: falta de espaço em disco para
o download dos projetos, falha no redirecionamento após a adição do projeto,
queda dos serviços no servidor de produção, lentidão no processo e falhas de
processamento. A equipe do Mezuro esteve sempre a disposição para auxiliar na
explanação e resolução destas falhas, seja via canal no IRC, via e-mail ou via
\textit{issues} no Github do projeto. Estes problemas foram de fato detectados
pela equipe na análise dos logs e nos relatos feitos, e foram registrados nos
seguintes links:

\begin{itemize}
  \item \textbf{Capacidade de Disco}:
    \begin{itemize}
      \item Handle full disk: \footnote{\url{https://github.com/mezuro/kalibro\_processor/issues/203}}
      \item Schedule removal job for repositories with no periodicity:
            \footnote{\url{https://github.com/mezuro/kalibro\_processor/issues/204}}
      \item Store SCM reference on a given processing:
            \footnote{\url{https://github.com/mezuro/kalibro\_processor/issues/205}}
      \item Is the aggregation working correctly?:
            \footnote{\url{https://github.com/mezuro/kalibro\_processor/issues/206}}
    \end{itemize}
  \item \textbf{Lentidão de Processamento}:
    \begin{itemize}
      \item Slow processings: \footnote{\url{https://github.com/mezuro/kalibro\_processor/issues/207}}
      \item Multiple calls to process trigger multiple redundant periodic processings:
            \footnote{\url{https://github.com/mezuro/kalibro\_processor/issues/208}}
      \item Separate processings in multiple queues:
            \footnote{\url{https://github.com/mezuro/kalibro\_processor/issues/209}}
    \end{itemize}
  \item \textbf{Falhas de Processamento}:
    \begin{itemize}
      \item Handle `Metric configuration has already been taken` error:
            \footnote{\url{https://github.com/mezuro/kalibro\_processor/issues/210}}
      \item Failed processing: \footnote{\url{https://github.com/mezuro/kolekti\_radon/issues/3}}
      \item Docker runner timeout: \footnote{\url{https://github.com/mezuro/kolekti\_cc\_phpmd/issues/11}}
    \end{itemize}
\end{itemize}

No CodeClimate, 16 softwares realizaram a \textit{build} da avaliação com
sucesso. Nos que falharam, alguns dos erros foram encontrados em três das
\textit{engines}: ora em \textit{duplication}, ora na \textit{phpmd}, ora na
\textit{eslint}. E um dos processos gerou o erro 500. Em todas as falhas, é
sugerido ao usuário entrar em contato com o suporte da ferramenta. As falhas
são relacionadas com erros de \textit{time out}. A sugestão é a exclusão de
arquivos com bibliotecas de terceiros, \textit{assets} de produção (como
arquivos minimizados) e talvez arquivos de testes automatizados.

% TODO: Resumo/discussão dos projetos de SPB

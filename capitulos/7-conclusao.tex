\chapter[Conclusão]{Conclusão}\label{chap:conclusao}

% TODO: refatorar conclusão como uma espécie de resposta à introduação

Os vários desafios na análise exploratória levaram à conclusões quanto à
aplicação de visualização de software no Mezuro. A primeira abordagem deste
trabalho seria a aplicação imediata, porém, observou-se que algumas pequenas
modificações poderiam atingir o mesmo objetivo e possivelmente atrair mais
usuários, fazendo com que a ferramenta saia do estado ``beta''. Em resumo, não
é recomendado à equipe de desenvolvimento o esforço de modificar a exibição da
avaliação do código-fonte com VS, em um primeiro momento, e sim modificar as
disposição dos itens já presentes na ferramenta.

Assim, recomenda-se para a evolução do Mezuro (i) a adaptação dos resultados da
análise exploratória para a criação da visualização dos softwares
aferidos pelo mesmo; (ii) a adequação aos padrões estipulados pelos mantenedores da
ferramenta, para uma possível contribuição, e (iii) a generalização da utilização das
novas telas de apresentação dos resultados. 

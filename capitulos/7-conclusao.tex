\chapter[Conclusão]{Conclusão}\label{chap:conclusao}

Os vários desafios na análise exploratória levaram à conclusões quanto à
aplicação de visualização de software no Mezuro. A primeira abordagem deste
trabalho seria a aplicação imediata, porém, observou-se que algumas pequenas
modificações poderiam atingir o mesmo objetivo e possivelmente atrair mais
usuários, fazendo com que a ferramenta saia do estado ``beta''. Como dito no
Capítulo anterior, não é recomendado à equipe de desenvolvimento o esforço de
modificar a exibição da avaliação do código-fonte com VS, e sim modificar as
disposição dos itens já presentes na ferramenta.

Outra conclusão é o observado pelo aluno no primeiro conjunto de contribuições:
a análise dos softwares do SPB. A maturidade atual do Mezuro não é o maior
desafio encontrado. Pôde-se observar que grande parte dos softwares não
apresentam boa qualidade no seu desenvolvimento, devido ao simples fato de que
não foi possível realizar a coleta e interpretação das métricas. Porém, o
conjunto analisado não deve representar a totalidade do Portal, muito menos
retirar a importância social e econômica da iniciativa.

\section{Trabalhos Futuros}

Levando em consideração o proposto neste trabalho, é planejado realizar: a
adaptação dos resultados da análise exploratória do Mezuro para a
criação da visualização dos softwares aferidos pelo mesmo; a adequação aos
padrões estipulados pelos mantenedores da ferramenta, para uma possível
contribuição; e a generalização da utilização das novas telas de apresentação
dos resultados. É interessante também continuar os estudos teóricos e a expansão
do presente trabalho. Possivelmente a tentativa de realizar novas interações com
o usuário.

Com o levantamento feito na categorização dos softwares do SPB, um outro
trabalho futuro seria a notificação aos administradores de cada repositório da
importância do versionamento do código-fonte e o pedido de um esforço para este
fim. Mediante o sucesso nesta abordagem, a incorporação dos demais softwares no
Mezuro seria a ideal conclusão. Porém, a comunidade é relutante muitas das
vezes. Portanto, é recomendado que, para se alcançar este futuro objetivo, que
se faça a solicitação de mudança da Instrução Normativa \cite{santos2011in01}
para abranger a todos os softwares a impreterível adição dos códigos-fonte
versionados no sistema de controle Git.

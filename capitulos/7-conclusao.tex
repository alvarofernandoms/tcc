\chapter[Conclusão]{Conclusão}\label{chap:conclusao}

Os vários desafios na análise exploratória levaram à conclusões quanto à
aplicação de visualização de software no Mezuro. A primeira abordagem deste
trabalho seria a aplicação imediata, porém, observou-se que algumas pequenas
modificações poderiam atingir o mesmo objetivo e possivelmente atrair mais
usuários, fazendo com que a ferramenta saia do estado ``beta''. Em resumo, não
é recomendado à equipe de desenvolvimento o esforço de modificar a exibição da
avaliação do código-fonte com VS, em um primeiro momento, e sim modificar as
disposições dos itens já presentes na ferramenta.

Assim, recomenda-se para a evolução do Mezuro: (i) a adaptação dos resultados da
análise exploratória para a criação da visualização dos softwares aferidos pelo
mesmo; (ii) a adequação aos padrões estipulados pelos mantenedores da
ferramenta, para uma possível contribuição, e (iii) a generalização da
utilização das novas telas de apresentação dos resultados.

Em um contexto geral, é de grande valia para o aluno todo o embasamento teórico
necessário para a realização deste trabalho. O estudo pode ser condensado em sete
tópicos. São eles: o Mezuro e sua arquitetura; Visualização de Software; o
Portal do SPB e seu funcionamento; os conceitos de Desing Sprint; prototipação
de telas; Code Climate e suas \textit{engines}; e o estudo de métricas de
software.

O mapeamento dos softwares do SPB ressalta algumas dificuldades e limitações.
Os desenvolvedores, maioria das vezes, ainda são relutantes em utilizar métricas
de software e monitoramento do código-fonte como um dos passos necessários para
a elevação da qualidade dos softwares. Porém, para este trabalho, a utilização
destes softwares foi importante tanto para o mapeamento das melhorias de
Back-End (capacidade de disco, lentidão e falhas de processamento), quanto para
as melhorias de design propostas pelo aluno. Todo esse conjunto de considerações
está registrado como \textit{issues} na organização do Mezuro no Github. Foram
abertos aproximadamente 16 destes problemas, que já estão sendo classificados
e divididos em outras pequenas tarefas, podendo ser desenvolvido por toda
comunidade (mantenedores e envolvidos).

A avaliação de técnicas de VS no Mezuro e a consequente pesquisa realizada são
apenas experimentos. A continuidade desta abordagem pode levar em consideração
as visualizações selecionadas e as opiniões dos entrevistados, mesmo que essas
sejam poucas.

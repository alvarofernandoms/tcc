\chapter{Metodologia}

\section{Objetivos}

O objetivo deste capítulo é apresentar a metodologia utilizada para a
realização da pesquisa e contribuição tecnológica. A principal questão
a ser respondida é: como a visualização de software pode auxiliar na
interpretação das métricas coletas e calculadas por ferramentas de análise
de código?

Considerando esta questão, é importante ressaltar que existem incontáveis
destas ferramentas. Porém estas analisam apenas aspectos mui específicos
do software, e falham (algumas delas) em demostrar ao engenheiro de
qualidade a interpretação dos dados analisados \cite{deissenboeck2011}.

Neste trabalho serão unidas determinadas métricas (TODO: definir quais
métricas), para gerar uma ou várias visualizações que auxiliem o usuário
do Mezuro a ter uma melhor interpretação do resultado gerado por esta
ferramenta.

Como exemplo dessa união e para a construção da prova de conceito
(TODO: é esse mesmo o nome?), é proposto utilizar-se da decisão dos
colegas Renan Costa Filgueiras e Vinícius Vieira Meneses: foram escolhidas
três métricas para a configuração da visualização em radar: \textit{number
of methods} (código: npm), \textit{number of public methods} (código nom)
e \textit{total number of modules} (código: total\_modules); e a decisão
de utilizar a biblioteca Javascript \textit{D3.js - Data-Driven Documents}.
(TODO: adicionar referência)


\section{Questão de pesquisa e hipóteses}

Reforçando a questão de pesquisa (QP) apresentada na seção anterior,
foram levantadas estas outras a baixo:

\begin{itemize}
  \item QP1 - Como a visualização de software pode auxiliar na
  interpretação das métricas coletadas e calculadas pelo Mezuro?
  \item QP2 - Quais técnicas de visualização que melhor se adequam ao
  contexto do Mezuro, possibilitando a aplicação em qualquer contexto de
  configuração?
  \item QP3 - Como representar as diferentes perspectivas sobre o processo
  de desenvolvimento e o produto de software?
\end{itemize}

As hipóteses são:

\begin{itemize}
  \item H1 - 
  \item H2 - 
\end{itemize}

% TODO: elaborar e documentar as hipótees

\section{Trabalhos Relacionados}

\section{O Mezuro}

O projeto Mezuro é uma ferramenta para extração, análise e interpretação de
métricas de código-fonte. De uma forma geral, ele é dividido em duas partes:
para o processamento e para o cálculo é utilizado o Kalibro, que é um
\textit{webservice}; e o Prezento para a interface gráfica (uma aplicação
\textit{Web}) \cite{meirellesCibse2015}. Sob (TODO: "sob" ou "sobre") a licença
\textit{Affero General Public License version 3} (AGPLv3), o Mezuro permite que
o usuário criar, salvar e editar "configurações" que são um conjunto de
métricas escolhidas por este e uma "grupo de leitura" para provimento de uma
interface gráfica de um conjunto de leitura que tenha algum sentido quando
agrupadas, por exemplo: com a pontuação x, a métrica terá o "rótulo" "BOM" e
será atribuído à ela a cor amarela; com a pontuação x+2, a métrica terá o 
"rótulo" "ÓTIMO" e estará com a cor verde. As cores são para destacar a
interpretação e são definidas com valores hexadecimal \cite{camarinhaOSS2015}.

O Kalibro, citado no parágrafo anterior, foi inicialmente escrito em Java para
compor uma das ferramentas do projeto QualiPSo (\textit{Quality Platform for
Open Source Software}). Já possuía maioria das funcionalidades presentes na
versão atual do Mezuro. Uma delas, bastante prática, era a de fornecer apenas o
URL do código compactado em arquivo ZIP ou TARBALL, ou o link para as
aplicações de controle de versão em SVN ou GIT, mais a escolha de uma
configuração, para iniciar a análise \cite{camarinhaOSS2015}. Em 2013 o Mezuro
passou a ser reescrito em Ruby, com objetivo de manter na mesma tecnologia as
camadas da arquitetura. Mudança justificada também pela necessidade do
processamento dos cálculos e da análise. A carga de requisições, mais a
quantidade que o servidor possuía, faziam com que a versão original do Kalibro
ficasse debilitado em seu fluxo de execuções. Outra vantagem dessa reescrita é
a facilidade com que novos contribuidores puderam/poderão entender todo o
funcionamento dessa parte do Mezuro. Algumas funcionalidades foram eliminadas
por serem consideradas não essenciais. E grande parte da primeira versão do
Kalibro está presente na gema kalibro\_gem\footnote{\url{https://rubygems.org/gems/kalibro_gem}}
\cite{meirellesCibse2015}.
(TODO: a gem kalibro\_gem e a kalibro\_client são a mesma coisa? A kalibro\_gem
foi descontinuada?)

% Sobre o Prezento

O Kalibro foi construído, nos primórdios, como um plugin da rede social
Noosfero. Com a decisão de reescrevê-lo em Ruby, a antiga interface gráfica,
que era "aproveitada" do Noosfero (TODO: era mesmo?), foi também escrito nesta
tecnologia. O Prezento, escrito utilizando o \textit{framework} para
desenvolvimento de aplicações \textit{Web} Ruby on Rails\footnote{\url{http://guides.rubyonrails.org/getting_started.html}},
é a redesenha e atual interface.

% Sobre a arquitetura do Mezuro

(TODO: eu preciso falar do QualiPSo?)

O projeto QualiPSo iniciado em meados de 2007 que tinha como objetivo aprimorar
as práticas de desenvolvimento aberto à época para atingir o reconhecimento e
confiabilidade que o FOSS possui hoje na indústria \cite{messias2012}. 

\subsection{Arquitetura e principais funcionalidades do Mezuro}

\section{D3 - Data-Driven Documents}


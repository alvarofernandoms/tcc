\chapter{Metodologia}\label{metodologia}

\section{Objetivos}

O objetivo deste capítulo é apresentar a metodologia utilizada para a
realização da pesquisa e contribuição tecnológica. A principal questão
a ser respondida é: como a visualização de software pode auxiliar na
interpretação das métricas coletadas e calculadas por ferramentas de análise
de código?

Considerando esta questão, é importante ressaltar que existem incontáveis
destas ferramentas. Porém estas analisam apenas aspectos mui específicos
do software, e falham (algumas delas) em demostrar ao engenheiro de
qualidade a interpretação dos dados analisados \cite{deissenboeck2011quamoco}.

Neste trabalho serão unidas determinadas métricas, para gerar uma ou várias
visualizações que auxiliem o usuário do Mezuro a ter uma melhor interpretação do
resultado gerado por esta ferramenta. E será realizado duas análises
exploratórias: do funcinonamento prático do Mezuro e dos softwares do SPB.

Como exemplo dessa união e para a construção do exemplo de uso, é proposto
utilizar-se da decisão dos colegas \citeonline{filgueiras2014mezuro}:
foram escolhidas três métricas para a configuração da visualização em
radar: \textit{number of methods} (código: npm), \textit{number of public
methods} (código nom) e \textit{total number of modules} (código:
total\_modules); e a decisão de utilizar a biblioteca Javascript \textit{D3.js -
Data-Driven Documents} \cite{filgueiras2014mezuro}.

\section{Trabalhos Relacionados}

No artigo \textit{CodeCrawler – An Extensible and Language Independent 2D and
3D Software Visualization Tool}, \citeonline{lanza2005codecrawler}
desenvolveram uma ferramenta para visualização de software capaz de exibir
detalhes de uma forma polimétrica, que é capaz de mostrar informações como as
métricas do software, sua semântica e a relação entre as classes, por exemplo.
Segundo o autor, essa visualização pode ser aplicada em vários contextos e não
somente em aspectos de software. Idealmente, a ferramenta é capaz de gerar
visualizações para software com grande número de linhas código. Porém é levado
em consideração a integração das visualizações em IDEs e essa solução não é
explorada. A utilização da técnica em diferentes ferramentas também não é
discutida.

Os autores, \citeonline{dosvisualizacc}, do \textit{Visualização de Software
Como Suporte ao Desenvolvimento Centrado em Métricas Orientadas a Objetivos}
criaram um plugin escrito em C++ para visualização de métricas OO, que são
coletadas pela ferramenta Qt Creator\footnote{\url{http://www.qt.io/ide/}}. As
métricas definidas foram: média de atributos por métodos em uma classe (MAC);
quantidade de métodos por classe (QMC); tamanho dos métodos por classe (TMC);
quantidade de atributos por classe (QAC); e quantidade de métodos públicos
(QMC). Métricas essas relacionadas com a categoria \textbf{estrutura} da VS.
E nesse plugin os casos de uso ou funcionalidades
são: calcular métricas: QAC; QMC; MAM; TMC; QMP. Além da
possibilidade de calcular a média das métricas, obter detalhes das métricas e
obter ajuda sobre as métricas. Um dos pontos importantes destacados no
resultado é a observação de que é possível garantir certo nível de qualidade
utilizando essas métricas até mesmo para projetos pequenos, com duas ou três
classes.

Porém a ferramenta só gera visualizações para aplicações escritas em C++, que é
um fator limitante devido à incorporação do plugin à ferramenta Qt Creator. O
plugin calcula também apenas cinco métricas consideradas fundamentais para
avaliar a qualidade de um software. E o detalhamento de determinada classe e
uma visualização da evolução do projeto em uma linha do tempo também não foram
abordados neste trabalho.

O artigo dos autores \citeonline{ramosanalise}, cujo o título é \textit{Análise de
Métricas Estáticas para Sistemas JavaScript}, não trata necessariamente de
visualização de software, embora em várias figuras mostrem distribuições dos
projetos e seus respectivos valores das métricas escolhidas. Portanto,
destacamos a escolha da ferramenta
\textit{https://github.com/jared-stilwell/escomplex}\footnote{\url{https://github.com/jared-stilwell/escomplex}},
que analisa métricas de projetos escritos em JavaScript (linguagem que ainda
não é suportada para análise no Mezuro), e escolha das métricas que são: número
de módulos; linhas de código; número de parâmetros; complexidade ciclomática;
densidade de complexidade ciclomática; métricas de complexidade de Halstead; e
índice de manutenção.

É importante ressaltar também o artigo \textit{Understanding software evolution
using a combination of software visualization and software metrics}. Os
autores, \citeonline{Lanza02understandingsoftware}, ressaltam que há uma grande
quantidade de dados hoje em circulação nos sistemas informatizados. Consideram ser este um
dos maiores problemas nas pesquisas de evolução de um software, e em alguns
casos as várias versões destes sistemas precisam ser analisadas em paralelo.
Então, considerando esta complexidade, uma técnica que pode auxiliar na
abordagem destas pesquisas é a visualização de software. É comentado sobre a
capacidade humana de observação de múltiplos aspectos de um problema em
paralelo e sua ligação direta com a visualização e o auxílio que possivelmente
esta trará o entendimento do sistema.

A técnica visual utilizada foi a matriz de evolução: combina visualização de
software com suas métricas \cite{lanza2001evolution}. Esta técnica permite
observar classes que merecem atenção, por exemplo, por terem crescido ou
encolhido durante a vida útil do sistema. E a abordagem apresentada é
independente da linguagem, mesmo que em um dos estudos de caso tenha sido
analisado um software escrito em Smalltalk.

Dissertando um pouco mais sobre a técnica, \citeonline{Lanza02understandingsoftware}
mostram suas minúcias. Cada coluna representa as diferentes versões do
software. Cada linha representa uma classe. E nessa técnica há algo importante
para este trabalho de conclusão de curso, onde é desejável trabalhar com o
software e a evolução do mesmo durante certo tempo ou período.

Características que podem ser observadas na técnica:

\begin{itemize}
  \item Tamanho dos sistemas em termos de classes. Quanto mais linhas, maior o
  sistema;
  \item Classes adicionadas e removidas;
  \item Períodos de decréscimo, estabilização ou crescimento do sistema durante a
  sua ``evolução''
  \item Classes \textit{Dayfly}. Classes com um tempo de vida pequeno, por
  exemplo que apenas apareceram em uma única versão do sistema;
  \item Classes que se mantêm (o contrário da característica observada no ponto
  anterior).
\end{itemize}

Porém, no artigo foi constatado que essa representação não é precisa o suficiente.
A evolução então seria o uso das métricas de código-fonte.

A Figura \ref{fig:evolutionMatrixAspects} foi extraída do artigo e mostra essas
características ou aspectos na evolução de um sistema, utilizando a técnica da
matriz de evolução.

\begin{figure}[!htb]
  \centering
    \includegraphics[keepaspectratio=true,scale=0.5]
    {figuras/evolutionMatrixAspects.eps}
  \caption{Aspectos da evolução de um sistema utilizando Matriz de Evolução
  \cite{Lanza02understandingsoftware}}
  \label{fig:evolutionMatrixAspects}
\end{figure}


Ainda comentando sobre o artigo \textit{Understanding software evolution using
a combination of software visualization and software metrics}, destacamos
o modo como foi inserida esta evolução da utilização das métricas de código-fonte:
altura e largura das caixas na visualização representam métricas das
classes. Métricas utilizadas: número de métodos (\textit{NOM}) e número de
atributos (\textit{NOA}).

E com as métricas definidas, foi feita uma categorização das classes em:

\begin{itemize}
  \item \textit{Pulsar}: uma classe que cresce e diminui durante o ciclo de
  vida do sistema;
  \item \textit{Supernova}: classe que de repente cresce bastante;
  \item \textit{White Dwarf}: classe que era grande, mas por algum motivo as
  suas responsabilidades foram divididas em outras classes e esta diminui;
  \item \textit{Red Giant}: é uma classe considerada \textit{god class}
  \cite{riel1996object}, por muito tempo, ou seja: uma classe grande que
  implementa muitas funcionalidades;
  \item \textit{Idle}: classe constante, inativa, sem grandes modificações.
\end{itemize}

Concluindo, o artigo apresenta uma abordagem útil que talvez auxilie no
entendimento de um sistema. Porém é abordado no mesmo uma quantidade de
métricas pequena (embora significativa para o contexto). Outro ponto negativo é
que em sistemas mui grandes, a visualização pode esbarrar em limitações de
exibição em uma única tela. E a abordagem pode apresentar problemas com nomes
repetidos de classes (considerando que este pode ser ou não um erro no
desenvolvimento de um sistema).

\section{Questão Problema e Hipóteses}

Reforçando a questão problema (QP), foram levantadas estas outras abaixo:

\begin{itemize}
  \item QP1 - Como a visualização de software pode auxiliar na
  interpretação das métricas coletadas e calculadas pelo Mezuro?
  \item QP2 - Quais técnicas de visualização que melhor se adequam ao
  contexto do Mezuro, possibilitando a aplicação em qualquer contexto de
  configuração?
  \item QP3 - Como representar as diferentes perspectivas sobre o processo
  de desenvolvimento e o produto de software?
\end{itemize}

As hipóteses são:

\begin{itemize}
  \item H1 - Não haverá uma visualização única para todas as métricas.
  \item H2 - Algumas métricas poderão ser apresentadas de uma forma combinada
  em uma única representação/visualização.
\end{itemize}

% TODO: elaborar e documentar as hipóteses

\section{Análises Exploratórias}

A análise exploratória do funcionamento prático do Mezuro, com os softwares do
SPB, foi realizada da seguinte forma:

\begin{itemize}
  \item Categorização dos softwares que possuiam código-fonte versionado;
  \item Download e versionamento dos que apenas possuem o código compactado no SPB;
  \item Criação de um espelho no Github, em uma organização chamada \textbf{spb-metrics}
  \item Adição dos softwares no Mezuro, utilizando as configurações para novas
  linguagens avaliadas por esta ferramenta:
    \subitem Configuração do Mezuro para PHP: CodeClimate PHPMD\footnote{\url{http://mezuro.org/pt/kalibro\_configurations/18}}
    \subitem Configuração do Mezuro para Python: Python\footnote{\url{http://mezuro.org/pt/kalibro\_configurations/8}}
  \item Comparação direta com o CodeClimate
\end{itemize}

\section{Proposta de Evolução da Visualização}

A atividade de contribuição tecnológica tem como objetivo selecionar métricas
com um certo nível  de similaridade e importância quando unidas, e a exibição de
tais por meio de uma das técnicas de visualização estudadas. Esta exibição
poderá ser uma nova página ou passo dentro do fluxo percorrido pelo usuário ao
utilizar o Mezuro para o monitoramento do código, ou ainda estar contida nas
informações gerais dos repositórios registrados na plataforma.

As etapas que serão seguidas para elaboração desta atividade estão descritas
nos itens abaixo e a Figura \ref{fig:metodotologia_atividades} ilustra o
encadeamento destas etapas:

\begin{figure}[h]
  \centering
    \includegraphics[keepaspectratio=true,scale=0.5]
    {figuras/metodotologia_atividades.eps}
  \caption{Encadeamento das etapas da atividade de contribuição tecnológica}
  \label{fig:metodotologia_atividades}
\end{figure}

% TODO: mudar estas atividades

\begin{enumerate}
  \item \textbf{Selecionar Configurações}: Selecionar as métricas com maior nível de
  similaridade e afinidade com os objetivos de qualidade conhecidos.

  \item \textbf{Criar configuração}: Criar o contêiner no qual as métricas
  selecionadas estarão presentes. A configuração servirá para a associação aos
  projetos selecionados e a abstração de alto nível da visualização desejada.
  Seja essa criação em ambiente de desenvolvimento, seja no próprio Mezuro em
  produção.

  \item \textbf{Selecionar projetos}: Selecionar quais projetos que melhor se
  adequem às métricas selecionadas e/ou possivelmente forneçam dados
  interessantes e significativos para a geração da visualização. Por exemplo:
  se as métricas selecionadas forem específicas de terminada linguagem, os
  projetos devem necessariamente ter seus códigos com maioria da escrita
  nessa linguagem. O número ideal é de três projetos, podendo conter a
  combinação entre projetos que são reconhecidos por possuírem uma boa
  organização com outros que são reconhecidos por não possuirem determinado
  nível de qualidade.

  \item \textbf{Adicionar projetos para análise}: Adicionar projetos como um
  novo repositório para análise na plataforma Mezuro.

  \item \textbf{Realizar captura dos resultados}: Uma vez que os resultados
  forem interessantes e significativos, será feita a captura (parser) desses
  dados utilizando talvez as Ruby Gems Sinatra ou Seed\_dump.

  \item \textbf{Selecionar visualização}: Antes de gerar a visualização, será
  feito a escolha da melhor visualização dado o contexto, relevância e
  granularidade dos dados resultantes da análise. Levando em consideração também
  as técnicas estudadas e um número finito pré-estabelecidos de visualizações.

  \item \textbf{Gerar visualização}: utilizar biblioteca de criação de
  visualização para criação da representação alternativa dos dados da análise.

  \item \textbf{Retornar exibição ao usuário}: Nesta fase será elaborada o
  retorno da visualização ao usuário, como mencionado antes, seja em uma nova
  página ou nas informações gerais do repositório analisado.
\end{enumerate}

% TODO: talvez as etapas relacionadas ao mezuro possam estar em um único passo. Por exemplo a atividade de "Criar configuração" que são apenas alguns cliques, ou apenas um comando no console.

\section{Seleção das Métricas}

% TODO: preciso definir as métricas agora?

% TODO: selecionar métricas com certo nível de similaridade.

% TODO: explique que, devido à comodidade em já ter algumas configurações pré-estabelicidas
% no Mezuro, e dado as limitações de tempo e escopo, não poderei passar por toda uma
% "Seleção das Métricas"
% SOLUÇÃO: explicar quais são essas configurações presentes por default no Mezuro

\section{Seleção dos Projetos}

% TODO: escrever um pouco sobre os projetos para geração das VS.
% Já tenho utilizado a Engine de IJE do professor Edson Alves como parâmetro.
% O professor Paulo sugeriu o Colab, dado o grande número de contribuições que ele atingiu nos
% últimos meses por conta do SPB.

% TODO: Mudar metodologia para abranger a análise exploratória e como ela foi realizada:
%   - Prioridades na escolha dos projetos
%   - Categorização nas planilhas
%   - Disponibilizar o códio em algum  sistema de versionamento
%   - Adicionar à ferramentas: CodeClimate e Mezuro.

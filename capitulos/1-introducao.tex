\chapter{Introdução}

% Primeiros parágrafos devem conter de forma clara o que o autor pretende com
% relação aos aspectos científicos e técnicos.

Aspectos da qualidade não são claros em muitos softwares atualmente e existe
uma grande complexidade em entender o seu funcionamento. A visualização de
software pode auxiliar e ou facilitar o entendimento destes aspectos e pode
reduzir essa complexidade. Não faltam dados disponíveis publicamente para
auxiliar na geração desse entendimento: código-fonte, históricos de evolução do
repositório, conjuntos de testes, relatórios de erros e suas soluções, trocas
de mensagens entre membros do projeto em listas de e-mail, artigos escritos por
acadêmicos que buscam fama, entre outros \cite{messias2012}
\cite{benkler2006wealth}.

É preciso processar esses dados de forma correta. E considerando também que
desenvolvedores são cruciais para o sucesso de um projeto e os valores
numéricos dos vários locais com dados disponibilizados publicamente em que é
possível coletar os dados citados no parágrafo anterior, eventualmente podem não ter
uma boa apresentação. Os dados podem ser de pouco valor, pois são numerosos e
difíceis de avaliar. Uma possível solução: visualização de informação
\cite{messias2012}. Visualizações de informação são ``representações visuais e
interativas de dados, apoiadas por computador, utilizadas para amplificar a
aquisição de conhecimento e apoiar descobertas, tomadas de decisão e
explicações a partir de dados complexos'' \cite{card1999readings}. Um problema
da visualização de software é selecionar uma técnica, pois nenhuma técnica
específica funciona para todos os problemas.

% Subtítulo aqui?

O software livre apresenta certas vantagens em detrimento do software restrito.
Geralmente licenciado sob as definições da
\textit{Free Software Foundation}\footnote{\url{http://www.gnu.org/philosophy/free-sw.pt-br.html}},
ou da \textit{Open Source Initiative}\footnote{\url{https://opensource.org/docs/definition.html}},
o software livre e aberto (\textit{Free and Open Source Software} - FOSS) tem
seu código aberto e acessível ao público e garante ao programador certos
aspectos que podem simplificar o desenvolvimento de um novo software ou a
evolução de um outro onde o envolvimento é desejado, seja por questões de
melhoria, ética, personalização ou simplesmente pelo interesse pessoal
\cite{meirelles2013monitoramento}. E muitas outras características agregam
valor ao uso, desenvolvimento e estímulo às contribuições ao FOSS: o
estabelecimento de uma \textit{cultura livre}, valorizando o mérito de questões
técnicas inerentes ao desenvolvedor; reconhecimento de bons programadores
capazes de escrever bons códigos e outros programadores melhores ainda capazes de
mantê-los; reconhecimento do valor das pessoas envolvidas e interessadas, seja
usuários finais, seja testadores de versões instáveis de software; entre outras
características \cite{raymond1999cathedral}.

Metodologias ágeis estão fortemente ligadas ao desenvolvimento de FOSS e
possuem formas de trabalho semelhantes \cite{meirelles2013monitoramento}. Esse
vínculo é expresso majoritariamente pelos os valores \cite{beck2001manifesto}:

\begin{itemize}
  \item Indivíduos e interações são mais importantes que processos e
  ferramentas;
  \item Software em funcionamento é mais importante que documentação abrangente;
  \item Colaboração com o cliente (usuários) é mais importante que negociação
  de contratos;
  \item Responder às mudanças é mais importante que seguir um plano.
\end{itemize}

Métricas de código estão ligadas ao desenvolvimento, planejamento, custos,
produtividade e qualidade do software. Existem dois grandes conjuntos delas:
métricas tradicionais e métricas de orientação a objetos. São métricas
tradicionais: métricas de tamanho (ex. Linhas de Código - LOC), métricas de
complexidade (ex. Complexidade Ciclomática), métricas manutenibilidade, dentre
outras (Tamanho Médio dos Módulos, Uso de Variável, Número de Funções, por
exemplo). E são métricas de orientação a objetos: métricas de classes (ex.
Encapsulamento dos Atributos), métricas de métodos (ex. Média de Complexidade
dos Métodos), métricas de acoplamento (ex. Acoplamento Aferente), métricas de
herança (ex. Medida de Polimorfismo) e métricas de sistema (ex. Reúso de
Classe) \cite{meirelles2013monitoramento}.

% Introdução com forte embasamento em Métodos Ágeis e FOSS, e Visualização pra apoiar
% citar PRMM, Fábio Kon... (verificar referências do Messias)

\section{Contexto}

Neste trabalho de conclusão de curso são apresentadas as relações entre
visualização de software, FOSS e metodologias ágeis com as devidas adaptações e
métricas de código-fonte para observação de que o entendimento do software e a
averiguação da qualidade do mesmo são passiveis através da visualização de
software. Para tanto, é proposto a adequação de técnicas usualmente utilizadas
para a geração de visualizações da informação às ferramentas de monitoramento
estático de código-fonte, em especial ao
Mezuro\footnote{\url{http://mezuro.org/}}.

\section{Objetivos}

Inicialmente a proposta deste trabalho é a geração de visualizações que podem
ser \textit{instanciadas} em determinadas ferramentas de monitoramento de
código-fonte, dada as adaptações para a coleta dos resultados da mesma. Com o
foco no Mezuro, o objetivo deste trabalho é investigar e utilizar-se da
visualização de software para auxiliar o controle da qualidade de um software.
É feito a união de determinadas métricas para gerar uma ou várias visualizações
que auxiliem o usuário a ter uma melhor interpretação do resultado gerado por
esta ferramenta.

\section{Organização do Trabalho}

O trabalho está organizado da seguinte maneira: Capítulo 2, onde será tratada a
VS como principal base para o referencial teórico e revisão bibliográfica;
Capítulo 3, que trata da metodologia para o desenvolvimento do mesmo; no
Capítulo 4 são apresentados os resultados preliminares com explicação do
principal exemplo de uso; e o Capítulo 5 com as conclusões, trabalhos futuros e
cronograma de atividades.

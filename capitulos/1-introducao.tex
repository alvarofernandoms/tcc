\chapter{Introdução}

% Primeiros parágrafos devem conter de forma clara o que o autor pretende com
% relação aos aspectos científicos e técnicos.

Aspectos da qualidade não são claros em muitos softwares atualmente e existe
uma grande complexidade em entender o seu funcionamento. A visualização de
software pode auxiliar e ou facilitar o entendimento destes aspectos e pode
reduzir essa complexidade. Não faltam dados disponíveis publicamente para
auxiliar na geração desse entendimento: código-fonte, históricos de evolução do
repositório, conjuntos de testes, relatórios de erros e suas soluções, trocas
de mensagens entre membros do projeto em listas de e-mail, artigos escritos por
acadêmicos que buscam fama, entre outros \cite{messias2012}
\cite{benkler2006wealth}.

É preciso processar esses dados de forma correta. E considerando também que
desenvolvedores são cruciais para o sucesso de um projeto e os valores
numéricos dos vários locais com dados disponibilizados publicamente em que é
possível coletar os dados citados no parágrafo anterior, eventualmente podem não ter
uma boa apresentação. Os dados podem ser de pouco valor, pois são numerosos e
difíceis de avaliar. Uma possível solução: visualização de informação
\cite{messias2012}. Visualizações de informação são ``representações visuais e
interativas de dados, apoiadas por computador, utilizadas para amplificar a
aquisição de conhecimento e apoiar descobertas, tomadas de decisão e
explicações a partir de dados complexos'' \cite{card1999readings}. Um problema
da Visualização de Software (VS) é selecionar uma técnica, pois nenhuma técnica
específica funciona para todos os problemas.

% Subtítulo aqui?

O software livre apresenta certas vantagens em detrimento do software restrito.
Geralmente licenciado sob as definições da
\textit{Free Software Foundation}\footnote{\url{http://www.gnu.org/philosophy/free-sw.pt-br.html}},
ou da \textit{Open Source Initiative}\footnote{\url{https://opensource.org/docs/definition.html}},
o software livre e aberto (\textit{Free and Open Source Software} - FOSS) tem
seu código aberto e acessível ao público e garante ao programador certos
aspectos que podem simplificar o desenvolvimento de um novo software ou a
evolução de um outro onde o envolvimento é desejado, seja por questões de
melhoria, ética, personalização ou simplesmente pelo interesse pessoal
\cite{meirelles2013monitoramento}. E muitas outras características agregam
valor ao uso, desenvolvimento e estímulo às contribuições ao FOSS: o
estabelecimento de uma \textit{cultura livre}, valorizando o mérito de questões
técnicas inerentes ao desenvolvedor; reconhecimento de bons programadores
capazes de escrever bons códigos e outros programadores melhores ainda capazes de
mantê-los; reconhecimento do valor das pessoas envolvidas e interessadas, seja
usuários finais, seja testadores de versões instáveis de software; entre outras
características \cite{raymond1999cathedral}.

Metodologias ágeis estão fortemente ligadas ao desenvolvimento de FOSS e
possuem formas de trabalho semelhantes \cite{meirelles2013monitoramento}. Esse
vínculo é expresso majoritariamente pelos os valores \cite{beck2001manifesto}:

\begin{itemize}
  \item Indivíduos e interações são mais importantes que processos e
  ferramentas;
  \item Software em funcionamento é mais importante que documentação abrangente;
  \item Colaboração com o cliente (usuários) é mais importante que negociação
  de contratos;
  \item Responder às mudanças é mais importante que seguir um plano.
\end{itemize}

% Introdução com forte embasamento em Métodos Ágeis e FOSS, e Visualização pra apoiar
% citar PRMM, Fábio Kon... (verificar referências do Messias)

\section{Contexto}

Neste trabalho de conclusão de curso são apresentadas as relações entre
visualização de software, FOSS e metodologias ágeis com as devidas adaptações e
métricas de código-fonte para observação de que o entendimento do software e a
averiguação da qualidade do mesmo são passiveis através da visualização de
software. Para tanto, é proposto a análise exploratória do projeto
Mezuro\footnote{\url{http://mezuro.org/}} que é uma ferramenta de monitoramento
de código-fonte. E também a adequação de técnicas usualmente utilizadas
para a geração de visualizações das informações geradas a partir da análise.

Esta análise exploratória tem como motivação e visa identificar motivos para o
baixo número de usuários ativos no Mezuro, considerando que a aplicação não
possui visualização de software, sendo este um dos possíveis motivos para a
baixa popularidade. Uma aplicação que serviu como comparação foi o
CodeClimate\footnote{\url{https://codeclimate.com/}}.

As contribuições idealizadas são portanto: análise exploratória do uso do Mezuro,
uma vez que possui poucos usuários; validar as configurações do Mezuro na
análise e interpretação de métricas; análise dos projetos do Portal do Software
Público Brasileiro (SPB); e especificações da VS no Mezuro.

\section{Objetivos}

Inicialmente a proposta deste trabalho é explorar todo funcionamento do Mezuro,
com foco nos coletores adicionados recentemente para projetos desenvolvidos nas
linguagens de programação PHP e Python. Esta análise servirá para guiar suas próprias evoluções. Os softwares que foram utilizados como
exemplos de uso são os do Portal do Software Público Brasileiro (SPB)
categorizados como Software Público de fato, de acordo com os requisitos
técnicos e jurídicos estabelecidos na IN 01/2011 \cite{santos2011in01}.

Posteriormente, a geração de visualizações que podem
ser \textit{instanciadas} em determinadas ferramentas de monitoramento de
código-fonte, dada as adaptações para a coleta dos resultados da mesma. Com o
foco no Mezuro, o objetivo deste trabalho é investigar e utilizar-se da
visualização de software para auxiliar o controle da qualidade de um software.
É feito a união de determinadas métricas para gerar uma ou várias visualizações
que auxiliem o usuário a ter uma melhor interpretação do resultado gerado por
esta ferramenta.

\section{Organização do Trabalho}

O trabalho está organizado da seguinte maneira: capítulo 2, onde será tratado o
Mezuro como principal ferramenta, foco da análise exploratória; capítulo 3, onde
será tratada a Visualização de Software como principal base para o
referencial teórico e revisão bibliográfica do estudo sobre VS; capítulo 4,
que trata da metodologia para o desenvolvimento do mesmo; no capítulo 5 é
apresentada a análise das métricas de código-fonte dos projetos do SPB via
Mezuro, com explicação de um dos casos de uso; capítulo 6, com a proposta de VS
para o Mezuro; e o capítulo 7 com as conclusões e trabalhos futuros;

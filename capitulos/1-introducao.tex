\chapter{Introdução}

% TODO: SUPER TAREFA! Organizar imagens/figuras antes dos textos que as explicam

Aspectos da qualidade não são claros em muitos softwares atualmente e existe
uma grande complexidade em entender o seu funcionamento. A visualização de
software pode auxiliar e ou facilitar o entendimento destes aspectos e pode
reduzir essa complexidade. Não faltam dados disponíveis publicamente para
auxiliar na geração desse entendimento: código-fonte, históricos de evolução do
repositório, conjuntos de testes, relatórios de erros e suas soluções, trocas
de mensagens entre membros do projeto em listas de e-mail, artigos escritos por
acadêmicos que buscam fama, entre outros \cite{messias2012}
\cite{benkler2006wealth}.

É preciso processar esses dados de forma correta. E considerando também que
desenvolvedores são cruciais para o sucesso de um projeto e os valores
numéricos dos vários locais com dados disponibilizados publicamente em que é
possível coletar os dados citados no parágrafo anterior, eventualmente podem não ter
uma boa apresentação. Os dados podem ser de pouco valor, pois são numerosos e
difíceis de avaliar. Uma possível solução: visualização de informação
\cite{messias2012}. Visualizações de informação é toda aquela representação
desenvolvida por computador que tem como objetivo a geração e aplicação de
conhecimento. Podem ser interativas ou não, e podem influenciar analistas de
dados nas tomadas de decisões e na transmissão desses conhecimentos a partir de
dados complexos \cite{card1999readings}. Um problema da Visualização de Software
(VS) é selecionar uma técnica, pois nenhuma técnica específica funciona para
todos os problemas.

O software livre apresenta certas vantagens em detrimento do software restrito.
Geralmente licenciado sob as definições da
\textit{Free Software Foundation}\footnote{\url{http://www.gnu.org/philosophy/free-sw.pt-br.html}},
ou da \textit{Open Source Initiative}\footnote{\url{https://opensource.org/docs/definition.html}},
o software livre e aberto (\textit{Free and Open Source Software} - FOSS) tem
seu código aberto e acessível ao público e garante ao programador certos
aspectos que podem simplificar o desenvolvimento de um novo software ou a
evolução de um outro onde o envolvimento é desejado, seja por questões de
melhoria, ética, personalização ou simplesmente pelo interesse pessoal
\cite{meirelles2013monitoramento}. E muitas outras características agregam
valor ao uso, desenvolvimento e estímulo às contribuições ao FOSS: o
estabelecimento de uma \textit{cultura livre}, valorizando o mérito de questões
técnicas inerentes ao desenvolvedor; reconhecimento de bons programadores
capazes de escrever bons códigos e outros programadores melhores ainda capazes de
mantê-los; reconhecimento do valor das pessoas envolvidas e interessadas, seja
usuários finais, seja testadores de versões instáveis de software; entre outras
características \cite{raymond1999cathedral}.

Neste trabalho, é apresentada uma análise exploratória da plataforma \textbf{livre}
Mezuro\footnote{\url{http://mezuro.org/}}, que é uma ferramenta de monitoramento
de código-fonte. O primeiro objetivo pensado para esta análise foi verificar a
necessidade de aplicação de técnicas de Visualização de Software na plataforma
Mezuro. Isso porque, é estabelicido como motivação identificar os motivos do
baixo número de usuários e número de projetos no Mezuro, em comparação, por
exemplo, com o Code Climate\footnote{\url{https://codeclimate.com/}}, que foi
desenvolvido posteriormente ao Mezuro.

Para realizar a análise exploratória proposta, inicialmente foi necessário
validar uma das principais funcionalidades do Mezuro que são as configurações
para a intepretação dos valores de métricas. Isso porque, os exemplos de uso
selecionados para esta análise foram os projetos do Portal do Software
Público Brasileiro (SPB), em que estava previsto para a incorporação do Mezuro.

\section{Objetivos}

O objetivo deste trabalho é avaliação da exibição dos dados das métricas de
código-fonte, com intuito de apontar evoluções para o Mezuro deixar de ser uma
plataforma em estado ``Beta''.
%
Isto também com o objetivo de explorar todo funcionamento do Mezuro, em especial,
com foco nos coletores adicionados recentemente para projetos desenvolvidos nas
linguagens de programação PHP e Python, no contexto da avaliação dos projetos do
Portal do Software Público Brasileiro.
%
Assim gerando insumos para comparação do Mezuro com outras plataformas de
avaliação de métricas de código-fonte.

\section{Organização do Trabalho}

% TODO: alterar esta organização

O trabalho está organizado da seguinte maneira: capítulo 2, onde será tratado o
Mezuro como principal ferramenta, foco da análise exploratória; capítulo 3, onde
será tratada a Visualização de Software como principal base para o
referencial teórico e revisão bibliográfica do estudo sobre VS; capítulo 4,
que trata da metodologia para o desenvolvimento do mesmo; no capítulo 5 é
apresentada a análise das métricas de código-fonte dos projetos do SPB via
Mezuro, com explicação de um dos casos de uso; capítulo 6, com a proposta de VS
para o Mezuro; e o capítulo 7 com as conclusões e trabalhos futuros;

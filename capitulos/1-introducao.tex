\chapter{Introdução}

% TODO: SUPER TAREFA! Organizar imagens/figuras antes dos textos que as explicam

A chamada ``crise do software'' \cite{arthur1985measuring} foi a grande
responsável pela necessidade de averiguação e evolução da qualidade do produto
da engenharia de software. Os requisitos que definem a qualidade de um sistema
e também estão envolvidos com a sua complexidade são
\cite{behkamal2009customizing}: execução correta das funcionalidades previstas;
confiabilidade; usabilidade; eficiência; manutenibilidade; e portabilidade. Para
o desenvolvimento de um software, são levados em consideração aspectos como
tamanho e interação entre os diversos componentes \cite{koscianski2007qualidade},
podendo não seguir as diretrizes de qualidade da ISO 9126 de
\citeonline{behkamal2009customizing}.

Aspectos da qualidade não são claros em muitos softwares e existe
uma grande complexidade em entender o seu funcionamento. A visualização de
software pode auxiliar e ou facilitar o entendimento destes aspectos e pode
reduzir essa complexidade \cite{messias2012} \cite{benkler2006wealth}. Não
faltam dados disponíveis publicamente para auxiliar na geração desse
entendimento: código-fonte, históricos de evolução do repositório, conjuntos de
testes, relatórios de erros e suas soluções, trocas de mensagens entre membros
do projeto em listas de e-mail, artigos escritos por acadêmicos que buscam fama,
entre outros \cite{messias2012} \cite{benkler2006wealth}. A visualização de
informação pode auxiliar na análise e interpreteção destes dados
\cite{messias2012}. Visualização de informação é toda aquela representação
desenvolvida por computador que tem como objetivo a geração e aplicação de
conhecimento \cite{card1999readings}. Podem ser interativas ou não, e podem
influenciar desenvolvedores nas tomadas de decisões e na transmissão desses
conhecimentos a partir de dados complexos \cite{card1999readings}.

O software livre apresenta certas vantagens em detrimento do software restrito.
Geralmente licenciado sob as definições da
\textit{Free Software Foundation}\footnote{\url{http://www.gnu.org/philosophy/free-sw.pt-br.html}},
ou da \textit{Open Source Initiative}\footnote{\url{https://opensource.org/docs/definition.html}},
o software livre e aberto (\textit{Free and Open Source Software} - FOSS) tem
seu código aberto e acessível ao público e garante ao programador certos
aspectos que podem simplificar o desenvolvimento de um novo software ou a
evolução de um outro onde o envolvimento é desejado, seja por questões de
melhoria, ética, personalização ou simplesmente pelo interesse pessoal
\cite{meirelles2013monitoramento}. E muitas outras características agregam
valor ao uso, desenvolvimento e estímulo às contribuições ao FOSS: o
estabelecimento de uma \textit{cultura livre}, valorizando o mérito de questões
técnicas inerentes ao desenvolvedor; reconhecimento de bons programadores
capazes de escrever bons códigos e outros programadores melhores ainda capazes de
mantê-los; reconhecimento do valor das pessoas envolvidas e interessadas, sejam
usuários finais, sejam testadores de versões instáveis do software; entre outras
características \cite{raymond1999cathedral}.

Neste trabalho, é apresentada uma análise exploratória da plataforma \textbf{livre}
Mezuro\footnote{\url{http://mezuro.org/}}, que é uma ferramenta de monitoramento
de código-fonte. Primeiramente foi-se pensado para esta análise a verificação da
necessidade de aplicação de técnicas de Visualização de Software na plataforma
Mezuro. Isso porque é estabelecido como motivação identificar os motivos do
baixo número de usuários e número de projetos no Mezuro, em comparação, por
exemplo, com o Code Climate\footnote{\url{https://codeclimate.com/}}, que foi
desenvolvido posteriormente ao Mezuro.

Para realizar a análise exploratória proposta, inicialmente foi necessário
validar uma das principais funcionalidades do Mezuro que são as configurações
para a intepretação dos valores de métricas. Isso porque, os exemplos de uso
selecionados para esta análise foram os projetos do Portal do Software
Público Brasileiro (SPB), em que estava previsto para a incorporação do Mezuro.

\section{Objetivos}

O objetivo deste trabalho é avaliação da exibição dos dados das métricas de
código-fonte, com intuito de apontar evoluções para o Mezuro deixar de ser uma
plataforma em estado ``Beta''.
%
Isto também com o objetivo de explorar todo funcionamento do Mezuro, em especial,
com foco nos coletores adicionados recentemente para projetos desenvolvidos nas
linguagens de programação PHP e Python, no contexto da avaliação dos projetos do
Portal do Software Público Brasileiro.
%
Assim gerando insumos para comparação do Mezuro com outras plataformas de
avaliação de métricas de código-fonte.

\section{Organização do Trabalho}

O trabalho está organizado da seguinte maneira: Capítulo \ref{chap:mezuro}, onde
será tratado o Mezuro como principal ferramenta, foco da análise exploratória;
Capítulo \ref{chap:visualizacao}, onde será tratada a Visualização de Software
como principal base para o referencial teórico e revisão bibliográfica do estudo
sobre VS; Capítulo \ref{chap:metodologia}, que trata da metodologia para o
desenvolvimento do mesmo; no Capítulo \ref{chap:analise_exploratoria} é
apresentada a análise exploratória de fato do Mezuro, com explicação de um dos
casos de uso; Capítulo \ref{chap:proposta}, com a proposta de melhoria para o
Mezuro; e o Capítulo \ref{chap:conclusao} com as conclusões.
